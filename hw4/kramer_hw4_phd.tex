\documentclass[11pt]{article}

% ==== PACKAGES ==== %
% \usepackage{fullpage}
\usepackage{amsmath,amssymb,amsthm}
\usepackage{epic}
\usepackage{eepic}
\usepackage{hyperref}
\usepackage{listings}
\usepackage{float}
\usepackage{graphicx}
\usepackage{fancyhdr}
\usepackage{color}
\usepackage{bbm}
\usepackage[letterpaper, margin=1in]{geometry}

% ==== MARGINS ==== %
% \pagestyle{empty}
% \setlength{\oddsidemargin}{0in}
% \setlength{\textwidth}{6.8in}
% \setlength{\textheight}{9.5in}

\pagestyle{fancy}
\fancyhf{}
\rhead{ASEN 5044}
\lhead{Homework 1}
\rfoot{Page \thepage}


\newtheorem*{solution*}{Solution}
\newtheorem{lemma}{Lemma}[section]
\newtheorem{theorem}[lemma]{Theorem}
\newtheorem{claim}[lemma]{Claim}
\newtheorem{definition}[lemma]{Definition}
\newtheorem{corollary}[lemma]{Corollary}
\lstset{moredelim=[is][\bfseries]{[*}{*]}}

% ==== DOCUMENT PROPER ==== %
\begin{document}

\thispagestyle{empty}

% --- Header Box --- %
\newlength{\boxlength}\setlength{\boxlength}{\textwidth}
\addtolength{\boxlength}{-4mm}

\begin{center}\framebox{\parbox{\boxlength}{\bf
      Statistical Estimation \hfill Homework 2\\
      ASEN 5044 Fall 2018 \hfill Due Date: Sep 20, 2018\\
      Name: Andrew Kramer \hfill PhD Student
}}
\end{center}

\section*{Exercise 1}
In the game of blackjack, the player is initially dealt two cards from a deck of ordinary playing cards. Without going into all the game's details, it is enough to know the best possible hand for a player to receive on the initial deal is a combination of an ace of any suit and any face card or ten. What is the probability that a player will be dealt this combination?

\subparagraph*{}
The number of ways a single player can be dealt a blackjack (assuming there is only one player) is ${4\choose1}{16\choose1} = 64$. So the number of events in our event space (which we'll call $A$) is 64. The total number of ways to deal 2 cards to that player is ${52\choose2} = 1326$. So, the number of events in the whole sample space is 1326. the probability of a single player being dealt a blackjack is therefore
\begin{equation*}
	\frac{N_A}{N} = \frac{{4\choose1}}{{16\choose1}} = \frac{64}{1326} = 0.048
\end{equation*}

\section*{Exercise 2}
Discrete and random variables $X$ and $Y$ can each take on integer values 1, 3, and 5. The joint probability table of $X$ and $Y$ is given below.

\begin{table}[h!]
  \begin{center}
    \caption{Your first table.}
    \label{tab:table1}
    \begin{tabular}{c|c|c|c} % <-- Alignments: 1st column left, 2nd middle and 3rd right, with vertical lines in between
      X & Y=1 & Y=3 & Y=5\\
      \hline
      1 & 1/18 & 1/18 & 1/18 \\
      3 & 1/18 & 1/18 & 1/6 \\
      5 & 1/18 & 1/6 & 1/3\\
    \end{tabular}
  \end{center}
\end{table}

\subsection*{Problem (a)}
Are the random variables $X$ and $Y$ independent?

\subparagraph*{}
No, the table clearly shows that the probability of $X$ taking on certain values changes depending on the value of $Y$. For instance $P(X=3|Y=3)\neq P(X=3|Y=5)$ If $X$ and $Y$ were independent then $P(X=x|Y=y)$ would be the same for all values of $y$.

\subsection*{Problem (b)}
Find the unconditional (marginal) probability $P(Y=5)$.

\subparagraph*{}
The marginal probability $P(Y=5) = P(Y=5|X=1) + P(Y=5|X=3) + P(Y=5|X=5) = (1/18) + (1/6) + (1/3) = (5/9)$.

\subsection*{Problem (c)}
What is the conditional probability $P(Y=5|X=3)$?

\subparagraph*{}
Generally $P(A|B)=\frac{P(A,B)}{P(B)}$. This means we need to calculate the marginal probability $P(X=3)=(1/18)+(1/18)+(1/6)=(5/18)$. With this information we can find
\begin{equation*}
	P(Y=5|X=3)=\frac{P(Y=5,X=3)}{P(X=3)}=\frac{1/6}{5/18}=\frac{3}{5}
\end{equation*}

\section*{Exercise 3}
Determine the value of $a$ in the function 
\begin{equation*}
	f_X(x) = \begin{cases}
			 	ax(1-x) & x\in[0,1] \\
			 	0 & \text{otherwise}
			 \end{cases}
\end{equation*}
so that $f_X(x)$ is a valid probability density function.

\subparagraph*{}
For $f_X(x)$ to be a valid probability density function $\int_{-\infty}^{\infty}f_X(x)dx$ must be equal to one.
\begin{align*}
	1 &= \int_{-\infty}^{\infty}f_X(x)dx \\ 
	&= \int_{0}^{1}ax(1-x)dx \\
	&= a\int_{0}^{1}(x-x^2)dx \\
	&= a\Big[\frac{x^2}{2} - \frac{x^3}{3}\Big|_0^1 \\
	&= a\Big(\frac{1}{2}-\frac{1}{3}\Big) \\
	1 &= \frac{a}{6} \\
	a &= 6
\end{align*}

\section*{Exercise 4}
The probability density function of an exponentially distributed random variable is defined as follows
\begin{equation*}
	f_X(x)=\begin{cases}
		   		ae^{-ax} & x>0 \\
		   		0 & x\leq0 
		   \end{cases}
\end{equation*}
where $a\geq0$.

\subsection*{Problem (a)}
Find the probability distribution function of an exponentially distributed random variable.

\subparagraph*{}
\begin{align*}
	P(a\leq X\leq b) &= \int_a^b f_X(x)dx \\
	&= \int_a^b ae^{-ax}dx \\
	&= -e^{-ax}|_a^b
\end{align*}
assuming $0<a\leq b$.

\subsection*{Problem (b)}
Find the mean of an exponentially distributed random variable.

\subparagraph*{}
\begin{align*}
	E[X] &= \int_{-\infty}^\infty xf_X(x)dx \\
	&= a\int_0^\infty xe^{-ax}dx \\
	&= a\Bigg[\Big(-\frac{x}{a}-\frac{1}{a^2}\Big)e^{-ax}\Bigg|_0^\infty \\
	&= \Bigg[\Big(-x-\frac{1}{a}\Big)e^{-ax}\Bigg|_0^\infty \\
	&= \Big(-\infty-\frac{1}{a}\Big)e^{-a\infty} + \frac{1}{a}e^0
\end{align*}
Because $e^{-\infty}=0$, $E[X]=\frac{1}{a}$.

\subsection*{Problem (c)}
Find the second moment of an exponentially distributed random variable.

\subparagraph*{}
\subparagraph*{}
\begin{align*}
	E[X^2] &= \int_{-\infty}^\infty x^2f_X(x)dx \\
	&= \int_0^\infty x^2ae^{-ax}dx \\
	&= \frac{1}{a^2}\int_0^\infty t^2e^-tdt \\
	&= \frac{1}{a^2}\Gamma(2+1)=\frac{2!}{a^2} \\
	&= \frac{2}{a^2}
\end{align*}

\subsection*{Problem (d)}
Find the variance of an exponentially distributed random variable.

\subparagraph*{}
\begin{align*}
	\sigma_x^2 &= E[X^2]-(E[X])^2)\\
	&= \frac{2}{a^2} - \Big(\frac{1}{a}\Big)^2 \\
	&= \frac{1}{a^2}
\end{align*}

\subsection*{Problem (e)}
What is the probability that an exponentially distributed random variable takes on a value within one standard deviation of its mean?

\subparagraph*{}
The standard deviation is $\sqrt{\frac{1}{a}}=\frac{1}{a}$, so we simply need to evaluate the probability distribution function from $0$ to $\frac{2}{a}$:
\begin{align*}
	P\Big(0<X\leq\frac{2}{a}\Big) &= 1-e^{-ax}\Big|_0^{2/a} \\
	&= (1-e^{-a\frac{2}{a}})-(1-e^0) \\
	&= 1-e^{-2}
\end{align*}
So the probability that $X$ falls within one standard deviation of the mean is a constant, unrelated to the value of $a$.

\section*{Exercise 5}

\section*{Exercise 6}

\section*{AQ 1}

\section*{AQ 2}

\end{document}
