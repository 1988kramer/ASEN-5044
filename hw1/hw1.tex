\documentclass[11pt]{article}

% ==== PACKAGES ==== %
% \usepackage{fullpage}
\usepackage{amsmath,amssymb,amsthm}
\usepackage{epic}
\usepackage{eepic}
\usepackage{hyperref}
\usepackage{listings}
\usepackage{float}
\usepackage{graphicx}
\usepackage{fancyhdr}
\usepackage{color}
\usepackage{bbm}
\usepackage[letterpaper, margin=1in]{geometry}

% ==== MARGINS ==== %
% \pagestyle{empty}
% \setlength{\oddsidemargin}{0in}
% \setlength{\textwidth}{6.8in}
% \setlength{\textheight}{9.5in}

\pagestyle{fancy}
\fancyhf{}
\rhead{ASEN 5044}
\lhead{Homework 1}
\rfoot{Page \thepage}


\newtheorem*{solution*}{Solution}
\newtheorem{lemma}{Lemma}[section]
\newtheorem{theorem}[lemma]{Theorem}
\newtheorem{claim}[lemma]{Claim}
\newtheorem{definition}[lemma]{Definition}
\newtheorem{corollary}[lemma]{Corollary}
\lstset{moredelim=[is][\bfseries]{[*}{*]}}

% ==== DOCUMENT PROPER ==== %
\begin{document}

\thispagestyle{empty}

% --- Header Box --- %
\newlength{\boxlength}\setlength{\boxlength}{\textwidth}
\addtolength{\boxlength}{-4mm}

\begin{center}\framebox{\parbox{\boxlength}{\bf
      Statistical Estimation \hfill Homework 1\\
      ASEN 5044 Fall 2018 \hfill Due Date: Sep 6, 2018\\
      Name: Andrew Kramer \hfill CU identitykey: ankr1041
}}
\end{center}




\section*{Exercise 1}

Compute determinants for the following matrices by hand and state whether each one is invertible

\subsection*{Problem (a)}

\subparagraph*{}

\begin{align*}
	|A| &= 1 \begin{vmatrix} 5 & 4 \\ 9 & 7 \end{vmatrix} - 2 \begin{vmatrix} 6 & 4 \\ 8 & 7 \end{vmatrix} + 3 \begin{vmatrix} 6 & 5 \\ 8 & 9 \end{vmatrix} \\
	&= 1(-1) - 2(10) + 3(14) \\
	&= 21
\end{align*}

Because $|A|$ is nonzero $A$ is invertible.

\subsection*{Problem (b)}

\subparagraph*{}

\begin{align*}
	|A| &= 11 \begin{vmatrix} 57 & 0 & 10 \\ 91 & 1 & 71 \\ 23 & 0 & 71 \end{vmatrix} - 26 \begin{vmatrix} 64 & 0 & 10 \\ 83 & 1 & 71 \\ 54 & 0 & 71 \end{vmatrix} \\
	&= 11(57(71) + 10(-23)) - 26(64(71)+10(-54)) \\
	&= 41978 - 109642 \\
	&= -67655
\end{align*}

Because $|A|$ is nonzero $A$ is invertible.

\subsection*{Problem (c)}

\subparagraph*{}

Because $A_1 = -2A_3$ (where $A_i$ refers to the $i^\text{th}$ column of A), the columns of A are not linearly independent. This means $A$ is not invertible and $|A| = 0$.

\subsection*{Problem (d)}

\subparagraph*{}

Because the determinant of an upper triangular matrix is simply the product of its diagonal elements:
\begin{align*}
	|A| &= 1 \times 8 \times 55 \times 233 \times 610 \\
	&= 62537200
\end{align*}

Because $|A|$ is nonzero $A$ is invertible.

\section*{Exercise 2}
Prove each of the following statements:
\subsection*{Problem (a)}
If $a$ and $b$ are non-zero $n\times 1$ vectors, then the matrix $ab^T$ has rank $1$.
\subparagraph*{}
Column $i$ of the outer product of $a$ and $b$ is simply the vector $a$ multiplied by the scalar $b_i$. This means that every column of $ab^T$ is a scalar multiple of $a$, so none of the columns of $ab^T$ are linearly independent. Thus, the rank of $ab^T$ is always one if both $a$ and $b$ are nonzero.
\subsection*{Problem (b)}
$\text{tr}(AB) = \text{tr}(BA)$ if $A$ is an $m\times n$ matrix and $B$ is $n\times m$.
\subparagraph*{}
The trace of $AB$ can be expressed as 
\begin{align*}
	\text{tr}(AB) &= \sum_{i=1}^n(AB)_{ii} \\
	&= \sum_{i=1}^n \sum_{j=1}^n a_{ij}b_{ji}
\end{align*} 
Similarly, the trace of $BA$ is
\begin{align*}
	\text{tr}(BA) &= \sum_{j=1}^n(BA)_{jj} \\
	&= \sum_{j=1}^n \sum_{i=1}^n b_{ji}a_{ij}
\end{align*}
Because $\sum_{i=1}^n \sum_{j=1}^n a_{ij}b_{ji}$ is equal to $\sum_{j=1}^n \sum_{i=1}^n b_{ji}a_{ij}$ we can conclude that $\text{tr}(AB) = \text{tr}(BA)$.
\subsection*{Problem (c)}
If $A$ is invertible then $|A^{-1}| = \frac{1}{|A|}$.
\subparagraph*{}
If we start with $|AB| = |BA| = |A|\ |B|$ and replace $B$ with $A^{-1}$ we find that 
\begin{align*}
	|A|\ |A^{-1}| &= |AA^{-1}| \\
	&= |I| \\
	&= 1
\end{align*}
Because $|A|\ |A^{-1}| = 1$ it must be true that $|A^{-1}| = \frac{1}{|A|}$. 

\section*{Exercise 3}
Consider the equations of motion for the coupled $2$ mass $3$ spring system like the one discussed in lecture. Find a set of $A,B,C,D$ matrices for the state vector definition,
\begin{equation*}
	x = [q_1-q_2,\ \dot{q}_1-\dot{q}_2,\ q_1+q_2,\ \dot{q}_1+\dot{q}_2]^T 
\end{equation*}
and for observations $y=[q_1,\ q_2]^T$ and inputs $u=[u_1,\ u_2]^T$.

\subparagraph{}
The following solution makes the assumption, as was done in lecture, that $m_1 = m_2 = 1\text{kg}$ and $k_1 = k_2 = k_3 = 1\text{N/m}$.

\begin{align*}
	\ddot{q}_1 &= -q_1 - q_1 + q_2 - u_1 \\
	&= -2q_1 + q_2 - u_1 \\
	\ddot{q}_2 &= -q_2 + q_1 - q_2 + u_1 + u_2 \\
	&= -2q_2 + q_1 + u_1 + u_2 \\
	\ddot{q}_1-\ddot{q}_2 &= (-2q_1+q_2-u_1) - (q_1-2q_2+u_1+u_2) \\
	&= -3q_1 + 3q_2 - 2u_1 - u_2 \\
	&= -3(q_1 - q_2) -2u_1 - u_2 \\
	\ddot{q}_1+\ddot{q}_2 &= (-2q_1+q_2-u_1) + (q_1-2q_2+u_1+u_2) \\
	&= -q_1 - q_2 + u_2 \\
	&= -(q_1 + q_2) + u_2 
\end{align*}

From these results we can construct our $A,B,C$, and $D$ matrices as follows:

\begin{equation*}
	A=\begin{bmatrix} 0&1&0&0\\-3&0&0&0 \\ 0&0&0&1 \\ 0&0&-1&0 \end{bmatrix}\quad B=\begin{bmatrix} 0&0 \\ -2&-1 \\ 0&0 \\ 0&1 \end{bmatrix}\quad C=\begin{bmatrix} \frac{1}{2}&0&\frac{1}{2}&0 \\ -\frac{1}{2}&0&-\frac{1}{2}&0 \end{bmatrix}\quad D=\begin{bmatrix} 0&0 \\ 0&0 \end{bmatrix}
\end{equation*}

\section*{Exercise 4}
The linearized equations of motion for an orbiting satellite spinning with nominal angular rate $p_o$ about the x axis are
\begin{align*}
	\Delta\dot{p} &= \frac{M_x}{I_x} \\
	\Delta\dot{q} &= \frac{p_0(I_x-I_z)\Delta r + M_y}{I_y} \\
	\Delta\dot{r} &= \frac{p_0(I_y-I_x)\Delta q + M_z}{I_z}
\end{align*}
where $I_x,I_y,$ and $I_z$ are the moments of inertia about the roll, pitch, and yaw axes; $M_x,M_y,$ and $M_z$ are the corresponding input torques; and $\Delta p,\Delta q,$ and $\Delta r$ are perturbations in rolling, pitching, and yawing rates from the linearization point.

\subsection*{Problem (a)}
Using the state vector $x = [\Delta p, \Delta q, \Delta r]^T$, input vector $u = [M_x,M_y,M_z]^T$, and output vector $y=x$, put this system into state space form.

\subparagraph{}
The state space form of the given system is as follows:
\begin{equation*}
	A = \begin{bmatrix} 
		0 & 0 & 0 \\
		0 & 0 & \frac{p_o(I_x-I_z)}{I_y} \\
		0 & \frac{p_o(I_y-I_x)}{I_z} & 0
	\end{bmatrix} \quad 
	B = \begin{bmatrix}
		\frac{1}{I_x} & 0 & 0 \\
		0 & \frac{1}{I_y} & 0 \\
		0 & 0 & \frac{1}{I_z}
	\end{bmatrix} \quad
	C = \mathbb{I}_3 \quad 
	D = 0_3
\end{equation*}

\subsection*{Problem (b)}
Use Matlab's \texttt{expm} function to compute the state transition matrix for this system assuming that $I_y=750\text{kg m}^3$, $I_z=1000\text{kg m}^3$, $I_x=500 \text{kg m}^3$, $p_0 = 20 \text{rad/s}$, and $\Delta t=0.1\text{s}$.

\subparagraph*{}
\begin{equation*}
	e^{A\Delta t} = \begin{bmatrix}
					1 & 0 & 0 \\
					0 & 588.703 & 416.275 \\
					0 & 832.55 & 588.703
					\end{bmatrix}
\end{equation*}

\subsection*{Problem (c)}
Use the state transition matrix to compute and plot the state time history for 5s, assuming zero inputs and assuming initial states $\Delta q(0) = 0.1\text{rad/s}$, and $\Delta p(0) = \Delta r = 0$. What can you say about the behavior of this system in terms of stability?

\subparagraph*{}
As figure (BLANK) below shows, given a small perturbation in $\Delta q$ of $0.1$ rad/s, the perturbations along all axes diverge rapidly from the initial conditions.

\end{document}
