\documentclass[11pt]{article}

% ==== PACKAGES ==== %
% \usepackage{fullpage}
\usepackage{amsmath,amssymb,amsthm}
\usepackage{epic}
\usepackage{eepic}
\usepackage{hyperref}
\usepackage{listings}
\usepackage{float}
\usepackage{graphicx}
\usepackage{fancyhdr}
\usepackage{color}
\usepackage{bbm}
\usepackage[letterpaper, margin=1in]{geometry}

% ==== MARGINS ==== %
% \pagestyle{empty}
% \setlength{\oddsidemargin}{0in}
% \setlength{\textwidth}{6.8in}
% \setlength{\textheight}{9.5in}

\pagestyle{fancy}
\fancyhf{}
\rhead{ASEN 5044}
\lhead{Homework 1}
\rfoot{Page \thepage}


\newtheorem*{solution*}{Solution}
\newtheorem{lemma}{Lemma}[section]
\newtheorem{theorem}[lemma]{Theorem}
\newtheorem{claim}[lemma]{Claim}
\newtheorem{definition}[lemma]{Definition}
\newtheorem{corollary}[lemma]{Corollary}
\lstset{moredelim=[is][\bfseries]{[*}{*]}}

% ==== DOCUMENT PROPER ==== %
\begin{document}

\thispagestyle{empty}

% --- Header Box --- %
\newlength{\boxlength}\setlength{\boxlength}{\textwidth}
\addtolength{\boxlength}{-4mm}

\begin{center}\framebox{\parbox{\boxlength}{\bf
      Statistical Estimation \hfill Homework 2\\
      ASEN 5044 Fall 2018 \hfill Due Date: Sep 20, 2018\\
      Name: Andrew Kramer \hfill PhD Student
}}
\end{center}

\section*{Exercise 1}
Consider the equations of motion for a unit mass subjected to an inverse square law fore field,
\begin{align*}
	\ddot{r} &= r \dot{\theta}^2 + \frac{k}{r^2} + u_1(t) \\
	\ddot{\theta} &= -\frac{2\dot{\theta}\dot{r}}{r} + \frac{1}{r}u_2(t)
\end{align*}
where $r$ represents the radius from the center of the force field, $\theta$ gives the angle with respect to a reference direction in the orbital plane, $k$ is a constant, and $u_1$ and $u_2$ represent radial and tanjectial thrusts, respectively. It is easily shown that for the initial conditions $r(0) = r_0$, $\theta(0) = 0$, $\dot{r}(0) = 0$, and $\dot{\theta}(0) = \omega_0$ with nominal thrusts $u_1(t) = 0$ and $u_2(t) = 0$ for all $t \geq 0$ the equations of motion have as a solution the circular orbit given by
\begin{align*}
	r(t) &= r_0 = \text{constant} \\
	\dot{\theta}(t) &= \omega_0 = \text{constant} = \sqrt{\frac{k}{r_0^3}} \\
	\theta{t} &= \omega_0t + \text{constant}
\end{align*}
\subsection*{Problem (a)}
Pick a state vector for this system and express the original nonlinear ODEs in standard nonlinear state space form.

\subparagraph*{}
If we choose $x = [r,\ \theta, \dot{r}, \dot{\theta}]^T$ as our state vector and $y = [r,\ \theta]^T$ as our observation vector then we can express the original ODEs as
\begin{align*}
	\dot{x} &= \begin{bmatrix} \dot{x_1} \\ \dot{x_2} \\ \dot{x_3} \\ \dot{x_4} \end{bmatrix} 
	= \begin{bmatrix} x_3 \\
				x_4 \\
				x_1x_4^2 - \frac{k}{x_1^2}+u_1(t) \\
				-\frac{2x_4x_3}{x_1} + \frac{1}{x_1}u_2(t)
				\end{bmatrix} \\
	y &= \begin{bmatrix} x_1 \\ x_2 \end{bmatrix}
\end{align*}

\subsection*{Problem (b)}
Linearize this system's nominal equations of motion about the nominal solution $r(t) = r_0$, $\dot{r}(0) = 0$, $\theta(t) = \omega_0t+\text{constant}$ and $\dot{\theta}(t) = \omega_0$ with $u_1(t) = 0$ and $u_2(t) = 0$. Find $(A,B,C,D)$ matrices for output $y(t)=[r(t),\theta(t)]^T$ for the linearized system of equations about the nominal solution.

\subparagraph*{}
If we take $x_\text{nom} = [r_0,0,\omega_0t+c,\omega_0]^T$ and $u_\text{nom} = [0,0]^T$. We can say $x(t)=x_\text{nom}(t)+\tilde{x}(t)$ and $u(t) = u_\text{nom}(t) + \tilde{u}(t)$. We can define 
\begin{align*}
	\dot{\tilde{x}} &= A_\text{nom} \tilde{x}(t) + B_\text{nom}\tilde{u}(t) \\
	A_\text{nom} = \left. \frac{\partial f}{\partial x} \right|_{x_\text{nom},u_\text{nom}} &= \begin{bmatrix} 0 & 0 & 1 & 0 \\
	0 & 0 & 0 & 1 \\ \omega_0^2+2\frac{k}{r_0^3} & 0 & 0 & 2r_0\omega_0 \\ 0 & 0 & -\frac{2\omega_0}{r_0} & 0 \end{bmatrix} \\
	B_\text{nom} = \left. \frac{\partial f}{\partial u} \right|_{x_\text{nom},u_\text{nom}} & = \begin{bmatrix} 0 & 0 \\ 0 & 0 \\ 1 & 0 \\ 0 & \frac{1}{r_0} \end{bmatrix}
\end{align*}
The observation function is already linear, so the $C$ and $D$ matrices do not need to be linearized:
\begin{align*}
	y(t) &= Cx(t) + Du(t) \\
	C &= \begin{bmatrix} 1 & 0 & 0 & 0 \\ 0 & 1 & 0 & 0 \end{bmatrix} \\
	D &= \begin{bmatrix} 0 & 0 \\ 0 & 0 \end{bmatrix}
\end{align*}

\subsection*{Problem (c)}
Convert the continuous time $(A,B,C,D)$ matrices you found from part (b) into discrete time $(F,G,H,M)$ matrices using a discretization step size of $\Delta t=10\text{s}$ and setting $k=398600\text{km}^3/\text{s}^2$ and $r_0=6678$km.

\subparagraph*{}
We start by reorganizing our ODE as $\dot{\tilde{x}}_a = \hat{A}[\tilde{x}, \tilde{u}]^T$ where
\begin{equation*}
	\hat{A} = \begin{bmatrix} A & B \\ 0 & 0 \end{bmatrix} = \begin{bmatrix} 0 & 0 & 1 & 0 & 0 & 0 \\ 0 & 0 & 0 & 1 & 0 & 0 \\ \omega_0^2 + 2\frac{k}{r_0^3} & 0 & 0 & 2r_0\omega_0 & 1 & 0 \\ 0 & 0 & -\frac{2\omega_0}{r_0} & 0 & 0 & \frac{1}{r_0} \\ 0 & 0 & 0 & 0 & 0 & 0 \\ 0 & 0 & 0 & 0 & 0 & 0 \end{bmatrix} = \begin{bmatrix} 0 & 0 & 1 & 0 & 0 & 0 \\ 0 & 0 & 0 & 1 & 0 & 0 \\ 0.00157 & 0 & 0 & 0.01788 & 1 & 0 \\ 0 & 0 & 4.01\times10^{-10} & 0 & 0 & 1.497\times10^{-4} \\ 0 & 0 & 0 & 0 & 0 & 0 \\ 0 & 0 & 0 & 0 & 0 & 0 \end{bmatrix}
\end{equation*}
We know that $F$ will be the upper left $n\times n$ submatrix and $G$ will be the the upper right $m\times n$ submatrix of $e^{\hat{A}\Delta t}$:
\begin{align*}
	e^{\hat{A}\Delta t} &= \begin{bmatrix} 1.079 & 0 & 10.26 & 0.906 & 50.65 & 4.496\times10^{-4} \\ 0 & 1 & 0 & 10 & 0 & 7.485\times10^{-3} \\ 0.0161 & 0 & 1.079 & 0.184 & 10.264 & 1.35\times10^{-4} \\ 0 & 0 & 0 & 1 & 0 & 1.497\times10^{-3} \\ 0 & 0 & 0 & 0 & 1 & 0 \\ 0 & 0 & 0 & 0 & 0 & 1 \end{bmatrix} \\
	F &= \begin{bmatrix} 1.079 & 0 & 10.26 & 0.906 \\ 0 & 1 & 0 & 10 \\ 0.0161 & 0 & 1.079 & 0.184\\ 0 & 0 & 0 & 1 \end{bmatrix} \\
	G &= \begin{bmatrix} 50.65 & 4.496\times10^{-4} \\ 0 & 7.485\times10^{-3} \\ 10.264 & 1.35\times10^{-4} \\ 0 & 1.497\times10^{-3} \end{bmatrix}
\end{align*}
The $H$ and $M$ matrices for the discretized system are simply equal to the $C$ and $D$ matrices for the continuous time system:
\begin{align*}
	H &= C \\
	M &= D
\end{align*}

\subsection*{Problem (d)}
Interpret the results for the STM in part (c), i.e. what is the physical meaning of each column vector that makes up $F$?

\subparagraph{}
Each column vector $F_i$ in $F$ represents how $\tilde{x}_i(k)$ contributes to $\tilde{x}(k+1)$. In other words, each column vector describes how the corresponding entry in the $\tilde{x}$ vector at step $k$ will affect the entire $\tilde{x}$ vector at step $k+1$.

\section*{Exercise 2}
The linear position $p$ of an object under constant acceleration is
\begin{equation*}
	p = p_0 + \dot{p}_0t + \frac{1}{2}\ddot{p}_0t^2
\end{equation*}
where $p_0$ is the initial position of the object.

\subsection*{Problem (a)}
Define a state vector as $x=[p\ \dot{p}\ \ddot{p}]^T$ and write the state space equation $\dot{x} = Ax$ for this system.

\subparagraph*{}
Because the object is under constant acceleration, the derivative of the acceleration $\dddot{p} = 0$. So the state space equation is simply:
\begin{equation*}
	\begin{bmatrix} \dot{p} \\ \ddot{p} \\ \dddot{p} \end{bmatrix} = \begin{bmatrix} 0 & 1 & 0 \\ 0 & 0 & 1 \\ 0 & 0 & 0 \end{bmatrix} \begin{bmatrix} p \\ \dot{p} \\ \ddot{p} \end{bmatrix}
\end{equation*}

\subsection*{Problem (b)}
Use all three expressions in Equation (1.71) to find the state transition matrix for this system.

\subparagraph*{}

\subsection*{Problem (c)}
Prove for the state transition matrix found above that $e^{A0} = I$.

\subparagraph*{}


\end{document}
