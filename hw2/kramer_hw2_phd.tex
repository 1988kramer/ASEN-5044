\documentclass[11pt]{article}

% ==== PACKAGES ==== %
% \usepackage{fullpage}
\usepackage{amsmath,amssymb,amsthm}
\usepackage{epic}
\usepackage{eepic}
\usepackage{hyperref}
\usepackage{listings}
\usepackage{float}
\usepackage{graphicx}
\usepackage{fancyhdr}
\usepackage{color}
\usepackage{bbm}
\usepackage[letterpaper, margin=1in]{geometry}

% ==== MARGINS ==== %
% \pagestyle{empty}
% \setlength{\oddsidemargin}{0in}
% \setlength{\textwidth}{6.8in}
% \setlength{\textheight}{9.5in}

\pagestyle{fancy}
\fancyhf{}
\rhead{ASEN 5044}
\lhead{Homework 1}
\rfoot{Page \thepage}


\newtheorem*{solution*}{Solution}
\newtheorem{lemma}{Lemma}[section]
\newtheorem{theorem}[lemma]{Theorem}
\newtheorem{claim}[lemma]{Claim}
\newtheorem{definition}[lemma]{Definition}
\newtheorem{corollary}[lemma]{Corollary}
\lstset{moredelim=[is][\bfseries]{[*}{*]}}

% ==== DOCUMENT PROPER ==== %
\begin{document}

\thispagestyle{empty}

% --- Header Box --- %
\newlength{\boxlength}\setlength{\boxlength}{\textwidth}
\addtolength{\boxlength}{-4mm}

\begin{center}\framebox{\parbox{\boxlength}{\bf
      Statistical Estimation \hfill Homework 2\\
      ASEN 5044 Fall 2018 \hfill Due Date: Sep 20, 2018\\
      Name: Andrew Kramer \hfill PhD Student
}}
\end{center}

\section*{Exercise 1}
Consider the equations of motion for a unit mass subjected to an inverse square law fore field,
\begin{align*}
	\ddot{r} &= r \dot{\theta}^2 + \frac{k}{r^2} + u_1(t) \\
	\ddot{\theta} &= -\frac{2\dot{\theta}\dot{r}}{r} + \frac{1}{r}u_2(t)
\end{align*}
where $r$ represents the radius from the center of the force field, $\theta$ gives the angle with respect to a reference direction in the orbital plane, $k$ is a constant, and $u_1$ and $u_2$ represent radial and tanjectial thrusts, respectively. It is easily shown that for the initial conditions $r(0) = r_0$, $\theta(0) = 0$, $\dot{r}(0) = 0$, and $\dot{\theta}(0) = \omega_0$ with nominal thrusts $u_1(t) = 0$ and $u_2(t) = 0$ for all $t \geq 0$ the equations of motion have as a solution the circular orbit given by
\begin{align*}
	r(t) &= r_0 = \text{constant} \\
	\dot{\theta}(t) &= \omega_0 = \text{constant} = \sqrt{\frac{k}{r_0^3}} \\
	\theta{t} &= \omega_0t + \text{constant}
\end{align*}
\subsection*{Problem (a)}
Pick a state vector for this system and express the original nonlinear ODEs in standard nonlinear state space form.

\subparagraph*{}
If we choose $x = [r,\ \theta, \dot{r}, \dot{\theta}]^T$ as our state vector and $y = [r,\ \theta]^T$ as our observation vector then we can express the original ODEs as
\begin{align*}
	\dot{x} &= \begin{bmatrix} \dot{x_1} \\ \dot{x_2} \\ \dot{x_3} \\ \dot{x_4} \end{bmatrix} 
	= \begin{bmatrix} x_3 \\
				x_4 \\
				x_1x_4^2 - \frac{k}{x_1^2}+u_1(t) \\
				-\frac{2x_4x_3}{x_1} + \frac{1}{x_1}u_2(t)
				\end{bmatrix} \\
	y &= \begin{bmatrix} x_1 \\ x_2 \end{bmatrix}
\end{align*}

\end{document}
