\documentclass[11pt]{article}

% ==== PACKAGES ==== %
% \usepackage{fullpage}
\usepackage{amsmath,amssymb,amsthm}
\usepackage{epic}
\usepackage{eepic}
\usepackage{hyperref}
\usepackage{listings}
\usepackage{float}
\usepackage{graphicx}
\usepackage{fancyhdr}
\usepackage{color}
\usepackage{bbm}
\usepackage[letterpaper, margin=1in]{geometry}

% ==== MARGINS ==== %
% \pagestyle{empty}
% \setlength{\oddsidemargin}{0in}
% \setlength{\textwidth}{6.8in}
% \setlength{\textheight}{9.5in}

\pagestyle{fancy}
\fancyhf{}
\rhead{ASEN 5044}
\lhead{Midterm 1}
\rfoot{Page \thepage}


\newtheorem*{solution*}{Solution}
\newtheorem{lemma}{Lemma}[section]
\newtheorem{theorem}[lemma]{Theorem}
\newtheorem{claim}[lemma]{Claim}
\newtheorem{definition}[lemma]{Definition}
\newtheorem{corollary}[lemma]{Corollary}
\lstset{moredelim=[is][\bfseries]{[*}{*]}}

% ==== DOCUMENT PROPER ==== %
\begin{document}

\thispagestyle{empty}

% --- Header Box --- %
\newlength{\boxlength}\setlength{\boxlength}{\textwidth}
\addtolength{\boxlength}{-4mm}

\begin{center}\framebox{\parbox{\boxlength}{\bf
      Statistical Estimation \hfill Midterm 1\\
      ASEN 5044 Fall 2018 \hfill Due Date: Oct 11, 2018\\
      Name: Andrew Kramer \hfill PhD Student
}}
\end{center}

\section*{Problem 1}
Inverted pendulum with equations of motion:
\begin{align*}
	(M+m)\ddot{z}-ml\ddot{\theta}\cos\theta&+ml\dot{\theta}^2\sin\theta=P \\
	l\ddot{\theta}-g\sin\theta &= \ddot{z}\cos\theta
\end{align*}
\subsection*{Part (a)}

The system's state equations can be expressed as follows:
\begin{equation*}
	\dot{x} = \begin{bmatrix} \dot{x}_1 \\ \dot{x}_2 \\ \dot{x}_3 \\ \dot{x}_4 \end{bmatrix} = \begin{bmatrix} x_2 \\ \frac{P-g\sin x_3 \cos x_3-mlx_4^2\sin x_3}{M+m\sin^2x_3}\cos x_3 \\ x_4 \\ \frac{P\cos x_3 + (M+m)g\sin x_3 + ml x_4^2\sin x_3 \cos x_3}{Ml + ml\sin^2x_3} \end{bmatrix}
\end{equation*}
To demonstrate the system is in equilibrium at $\dot{z}=0,\ \theta=0,\ \dot{\theta}=0$, and $P(t)=0$ we note first that at the given conditions the equations of motion become 
\begin{align*}
	(M+m)\ddot{z}-ml\ddot{\theta} &= 0 \\
	l\ddot{\theta} &= \ddot{z}
\end{align*}
If we plug the second equation back into the first we get
\begin{equation*}
	(M+m)\ddot{z} - m\ddot{z}=M\ddot{z}=0
\end{equation*}
Because we know $M$ is not equal to zero, this means $\ddot{z}$ must be equal to zero. Additionally, because $\ddot{z} = l\ddot{\theta}$ and $l\neq0$ we can also conclude that $\ddot{\theta}=0$. This means $\dot{x}=0$ under the given conditions and the system is therefore in equilibrium.

\subsection*{Part (b)}



\subsection*{Part (c)}
\subsection*{Part (d)}
\subsection*{Part (e)}
\subsection*{Part (f)}
\subsection*{Part (g)}
\section*{Problem 2}
Two 6-sided dice rolls with $R_1$ and $R_2$ denoting the outcome of the first and second die, respectively.

\subsection*{Part (a)}
$P(R_1)=P(R_2)=\frac{1}{6}$ for all $R_1$ and $R_2$. Because the outcomes $R_1$ and $R_2$ are independent, $P(R_1,R_2)=P(R_1)*P(R_2)$. So
\begin{equation*}
	P(R_1,R_2)=\frac{1}{36},\ \forall R_1,R_2
\end{equation*}

\subsection*{Part (b)}
The joint probabilities for $X$ and $Y$ are shown in table \ref{joints} below
\begin{table}[h!]
  \begin{center}
    \caption{Joint Probabilities}
    \label{joints}
    \begin{tabular}{c|c|c|c|c|c|c} % <-- Alignments: 1st column left, 2nd middle and 3rd right, with vertical lines in between
      X & Y=1 & Y=2 & Y=3 & Y=4 & Y=5 & Y=6 \\
      \hline
      1 & 1/36 & 2/36 & 2/36 & 2/36 & 2/36 & 2/36 \\
      2 & 0 & 1/36 & 2/36 & 2/36 & 2/36 & 2/36 \\
      3 & 0 & 0 & 1/36 & 2/36 & 2/36 & 2/36 \\
      4 & 0 & 0 & 0 & 1/36 & 2/36 & 2/36 \\
      5 & 0 & 0 & 0 & 0 & 1/36 & 2/36 \\
      6 & 0 & 0 & 0 & 0 & 0 & 1/36 \\
    \end{tabular}
  \end{center}
\end{table}

\subsection*{Part (c)}
The marginal probabilities of $X$ obtained from the sum $\sum_yP(X=x,Y=y)$
and the marginal probabilities of $Y$ obtained from the sum $\sum_xP(X=x,Y=y)$ are shown below in table \ref{marginals}.
\begin{table}[h!]
  \begin{center}
    \caption{Marginal Probabilities}
    \label{marginals}
    \begin{tabular}{c|c|c} % <-- Alignments: 1st column left, 2nd middle and 3rd right, with vertical lines in between
       & X & Y \\
      \hline
      1 & 11/36 & 1/36 \\
      2 & 9/36 & 3/36 \\
      3 & 7/36 & 5/36 \\
      4 & 5/36 & 7/36 \\
      5 & 3/36 & 9/36 \\
      6 & 1/36 & 11/36 \\
    \end{tabular}
  \end{center}
\end{table}

\subsection*{Part (d)}
$X$ and $Y$ are not independent. Two variables are considered independent if the realization of one variable does not affect the probability of the other. This is not the case for $X$ and $Y$. By the definitions of $X$ and $Y$, the value of $Y$ cannot be less than the value of $X$, since the maximum of $R_1$ and $R_2$ cannot be less than the minimum. So the $P(Y=3,X=5)=0$, while $P(Y=3,X=1) = 2/36$.

\section*{Problem 3}
A random variable $X$ has the pdf $p(x)=k(1-x^4)$ for $-1\leq x\leq1$ and $p(x)=0$ elsewhere.

\subsection*{Part (a)}
Because $\int_{-\infty}^\infty p(x)dx = 1$ we can find $k$ as follows:
\begin{align*}
	\int_{-1}^1 k(1-x^4)dx &= 1 \\
	k\int_{-1}^1(1-x^4)dx &= \\
	k \big[x-\frac{1}{5}x^5 \big|_{-1}^1 &= \\
	k\Big(1-\frac{1}{5}+1-\frac{1}{5}\Big) &= \\
	\frac{8k}{5} &= 1 \\
	k = \frac{5}{8}
\end{align*}
Now we can calculate $E[x] = \int_{-\infty}^\infty xp(x)dx$ as follows:
\begin{align*}
	\int_{-\infty}^\infty xp(x)dx &= \frac{5}{8}\int_{-1}^1x(1-x^4)dx \\
	&= \frac{5}{8}\int_{-1}^1(x-x^5)dx \\
	&= \frac{5}{8} \Bigg[\frac{1}{2}x^2 - \frac{1}{6}x^6 \Bigg|_{-1}^1 \\
	&= \frac{5}{8} \Bigg(\frac{1}{2} - \frac{1}{6} - \frac{1}{2} + \frac{1}{6}\Bigg) \\
	&= 0
\end{align*}
Next we find $E[x^2] = \int_{-\infty}^\infty x^2p(x)dx$ as
\begin{align*}
	\int_{-\infty}^\infty x^2p(x)dx &= \frac{5}{8}\int_{-1}^1x^2(1-x^4)dx \\
	&= \frac{5}{8}\int_{-1}^1(x^2-x^6)dx \\
	&= \frac{5}{8} \Bigg[\frac{1}{3}x^3 - \frac{1}{7}x^7 \Bigg|_{-1}^1 \\
	&= \frac{5}{8} \Bigg(\frac{1}{3} - \frac{1}{7} + \frac{1}{3} - \frac{1}{7}\Bigg) \\
	&= \frac{5}{21}
\end{align*}
Finally we can find $\text{var}(x) = E[x^2]-(E[x])^2 = \frac{5}{21}-0=\frac{5}{21}$.

\subsection*{Part (b)}
The cumulative distribution function is defined as $P_X(\zeta)=\int_{-\infty}^\zeta p(x)dx$. So the cdf is:
\begin{align*}
	P_X(\zeta)&=\int_{-\infty}^\zeta p(x)dx \\
	&= \frac{5}{8}\int_{-1}^\zeta (1-x^4)dx \\
	&= \frac{5}{8} \Big[x-\frac{1}{5}x^5\Big|_{-1}^\zeta \\
	&= \frac{5}{8} \Big(\zeta - \frac{1}{5}\zeta^5 + \frac{4}{5}\Big)
\end{align*}

\subsection*{Part (c)}
Because the pdf is symmetric about zero, $P(|X| < 0.5)$ is equivalent to $P(-0.5<X<0.5)$, which can be found as
\begin{equation*}
	P(-0.5<X<0.5) = P_X(0.5) - P_X(-0.5) = 0.7895
\end{equation*}

\section*{Problem 4}
Blood alcohol tests on drivers given the conditional probabilities given in table \ref{conditionals}:

\begin{table}[h!]
  \begin{center}
    \caption{Conditional Probabilities}
    \label{conditionals}
    \begin{tabular}{c|c|c} % <-- Alignments: 1st column left, 2nd middle and 3rd right, with vertical lines in between
      $P(T|A)$ & $A=\text{drunk}$ & $A=\text{sober}$ \\
      \hline
      $T=\text{positive}$ & 0.99 & 0.001 \\
      $T=\text{negative}$ & 0.01 & 0.999 \\
    \end{tabular}
  \end{center}
\end{table}

\subsection*{Part (a)}
We can find $P(A=\text{drunk}|T=\text{positive})$ through a straightforward application of Bayes' rule:
\begin{equation*}
	P(A=\text{drunk}|T=\text{positive}) = \frac{P(T=\text{positive}|A=\text{drunk})*P(A=\text{drunk})}{P(T=\text{positive})}
\end{equation*}
Since we aren't given a number for $P(T=\text{positive})$ we can find it as $P(T=\text{positive})=P(T=\text{positive}|A=\text{drunk})P(A=\text{drunk})+P(T=\text{positive}|A=\text{sober})P(A=\text{sober})=0.99*0.2+0.001*0.8=0.1988$. So the conditional probability is:
\begin{equation*}
	P(A=\text{drunk}|T=\text{positive}) = \frac{0.99*0.2}{0.1988} = 0.996
\end{equation*}
\subsection*{Part (b)}
If $P(A=\text{drunk})=0.001$ the probability $P(T=\text{positive})=0.99*0.001+.001*0.999=0.002$. So the conditional probability becomes
\begin{equation*}
	P(A=\text{drunk}|T=\text{positive}) = \frac{0.99*0.001}{0.002} = 0.495
\end{equation*}

\section*{Problem AQ1}
\subsection*{Part (a)}
\subsection*{Part (b)}
\subsection*{Part (c)}
\subsection*{Part (d)}
\end{document}
