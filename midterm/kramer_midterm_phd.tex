\documentclass[11pt]{article}

% ==== PACKAGES ==== %
% \usepackage{fullpage}
\usepackage{amsmath,amssymb,amsthm}
\usepackage{epic}
\usepackage{eepic}
\usepackage{hyperref}
\usepackage{listings}
\usepackage{float}
\usepackage{graphicx}
\usepackage{fancyhdr}
\usepackage{color}
\usepackage{bbm}
\usepackage[letterpaper, margin=1in]{geometry}

% ==== MARGINS ==== %
% \pagestyle{empty}
% \setlength{\oddsidemargin}{0in}
% \setlength{\textwidth}{6.8in}
% \setlength{\textheight}{9.5in}

\pagestyle{fancy}
\fancyhf{}
\rhead{ASEN 5044}
\lhead{Midterm 1}
\rfoot{Page \thepage}


\newtheorem*{solution*}{Solution}
\newtheorem{lemma}{Lemma}[section]
\newtheorem{theorem}[lemma]{Theorem}
\newtheorem{claim}[lemma]{Claim}
\newtheorem{definition}[lemma]{Definition}
\newtheorem{corollary}[lemma]{Corollary}
\lstset{moredelim=[is][\bfseries]{[*}{*]}}

% ==== DOCUMENT PROPER ==== %
\begin{document}

\thispagestyle{empty}

% --- Header Box --- %
\newlength{\boxlength}\setlength{\boxlength}{\textwidth}
\addtolength{\boxlength}{-4mm}

\begin{center}\framebox{\parbox{\boxlength}{\bf
      Statistical Estimation \hfill Midterm 1\\
      ASEN 5044 Fall 2018 \hfill Due Date: Oct 11, 2018\\
      Name: Andrew Kramer \hfill PhD Student
}}
\end{center}

\section*{Problem 1}
Inverted pendulum with equations of motion:
\begin{align*}
	(M+m)\ddot{z}-ml\ddot{\theta}\cos\theta&+ml\dot{\theta}^2\sin\theta=P \\
	l\ddot{\theta}-g\sin\theta &= \ddot{z}\cos\theta
\end{align*}
\subsection*{Part (a)}

The system's state equations can be expressed as follows:
\begin{equation*}
	\dot{x} = \begin{bmatrix} \dot{x}_1 \\ \dot{x}_2 \\ \dot{x}_3 \\ \dot{x}_4 \end{bmatrix} = \begin{bmatrix} x_2 \\ \frac{P-g\sin x_3 \cos x_3-mlx_4^2\sin x_3}{M+m\sin^2x_3}\cos x_3 \\ x_4 \\ \frac{P\cos x_3 + (M+m)g\sin x_3 + ml x_4^2\sin x_3 \cos x_3}{Ml + ml\sin^2x_3} \end{bmatrix}
\end{equation*}
To demonstrate the system is in equilibrium at $\dot{z}=0,\ \theta=0,\ \dot{\theta}=0$, and $P(t)=0$ we note first that at the given conditions the equations of motion become 
\begin{align*}
	(M+m)\ddot{z}-ml\ddot{\theta} &= 0 \\
	l\ddot{\theta} &= \ddot{z}
\end{align*}
If we plug the second equation back into the first we get
\begin{equation*}
	(M+m)\ddot{z} - m\ddot{z}=M\ddot{z}=0
\end{equation*}
Because we know $M$ is not equal to zero, this means $\ddot{z}$ must be equal to zero. Additionally, because $\ddot{z} = l\ddot{\theta}$ and $l\neq0$ we can also conclude that $\ddot{\theta}=0$. This means $\dot{x}=0$ under the given conditions and the system is therefore in equilibrium.

\subsection*{Part (b)}



\subsection*{Part (c)}
\subsection*{Part (d)}
\subsection*{Part (e)}
\subsection*{Part (f)}
\subsection*{Part (g)}
\section*{Problem 2}
\subsection*{Part (a)}
\subsection*{Part (b)}
\subsection*{Part (c)}
\subsection*{Part (d)}
\section*{Problem 3}
\subsection*{Part (a)}
\subsection*{Part (b)}
\subsection*{Part (c)}
\section*{Problem 4}
\subsection*{Part (a)}
\subsection*{Part (b)}
\section*{Problem AQ1}
\subsection*{Part (a)}
\subsection*{Part (b)}
\subsection*{Part (c)}
\subsection*{Part (d)}
\end{document}
